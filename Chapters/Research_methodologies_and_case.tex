\chapter{CASE \& RESEARCH METHODOLOGIES}\label{chapter:case_and_research_methodologies}
The specifics of the case, Hubshare, will be described in this chapter. After the introduction, the reasoning behind the research methods will be opened, and different research methods will be presented. Finally, the indicators for the success metrics of the research will be defined.

\section{Case: New test automation system for Hubshare}\label{section:case_new_test_automation_system_for_hubshare}
As described during the \autoref{section:context_of_the_project}, Hubshare is an M-Files subsidiary and a digital workplace portal software. During the initial development of the Hubshare product, Hubshare was a significantly smaller company compared to the current combined size of Hubshare and M-Files. Because the company was small, when the Hubshare product was introduced, all the development resources were mainly devoted to listening to customer wishes and adding new functionalities. Because of the strict focus on functionality and resource limitations, test automation was not considered at the beginning of the Hubshare product project. The resourcing issue is typical in the startup scene and has also been identified in the literature \cite{laukkanen2018comparison}.

Over the years, Hubshare's code base grew considerably more extensive. During the continuous development, test automation was considered from time to time, and all kinds of test automation trials and implementations were made. However, most of the test automation attempts were short-lived. Attempts dried out because the proper test automation processes were not established, and the technical things like \glsfirst{ci}-pipelines were left undone. After a while, half-completed test automation solutions were left without maintenance and slowly became unusable because the automation cases were not synchronized with the code changes.

At the beginning of this project, there were only two somewhat functional test automation systems. The first system implemented the unit tests for the \gls{aspnet} based backend. The system was fully functional, and even the \gls{ci}-pipeline was enabled for the system that the developers actively monitored. The problem with the existing system was that the test coverage was deficient, and the test cases tested almost exclusively trivial cases or, in other words, pure functions without any side effects. Another test automation system that was somewhat in use was \gls{api} integration level test automation. Test automation covered most of the publicly available \gls{api}. The problem with the \gls{api}-test automation system was that it was not functional in the \gls{ci}-pipeline, even though a pipeline did exist. Any test automation processes and plans were lacking in the beginning.

Even though Hubshare's initial situation was not ideal from the test automation perspective, there was still a silver lining at the starting point. Hubshare did have extensive and prolonged \gls{rat} testing in place, which assured the quality of the product and pruned away worst regression bugs. So in the initial situation, regression testing was mainly done manually, not automatically, but it was still being done and not ignored, which is the most important thing. The importance of regression testing has also been identified in the literature \cite{onoma1998regression}.

\section{Research methodologies}
During this section, methods that were used during the research will be presented. First, different research methods will be described shortly. After that, it will be explained why the specific methods were selected for this research.

\subsection{Case study}
Based on the \citeauthor{feagin1991case} definition, a case study is an in-depth investigation that studies a single social phenomenon using qualitative research methods \cite{feagin1991case}. Further, they define that the research should be conducted in great detail and that it often uses multiple data sources. According to \citeauthor{flyvbjerg2011case}, a case study mainly emphasises what to research instead of how to research it \cite{flyvbjerg2011case}. So defining clear boundaries for the research is vital during the case study, according to the book.

A case study was a natural research framework choice for this thesis because a single instance of the phenomenon is extensively studied during the research. As the case study requires, the subject under research is clearly defined. Even though the subject is broad, when viewed from a practical viewpoint, the subject can still be researched extensively. Extensive research is possible because the subject is restricted to higher-level management, introduction and architecture instead of considering the whole sociological and technical hierarchy from top to bottom. In addition, a multi-research approach fits this research well, which is encouraged while carrying out a case study.

\subsection{Semi-structured interviews}
Interviews are the most common way of collecting information among qualitative researchers \cite{cassell2005creating}. Overall there are four different forms of interviews: structured, semi-structured, unstructured and focus group interviews \cite{alsaawi2014critical}. During this research, semi-structured interviews were used. Semi-structured interviews have predefined questions, but unlike strictly structured interviews, they allow discussion to evolve into directions that were not planned \cite{alsaawi2014critical}. Semi-structured interviews often provide helpful information that a researcher could not anticipate before the interview \cite{raworth2012conducting}.

Initially, it was unclear what should be done and even researched. Interviews can provide in-depth qualitative information that will help understand the subject better. Interviews are helpful when questions are better answered in prose than with numbers, and there is a need to explore a trend \cite{beck2008practical}, which is the case in this research. Because of that, the interviews were selected as an information-gathering method for this thesis.

Specifically, semi-structured interviews were selected as the primary interviewing method because the other methods were not entirely fitting for the research. Generally, the software field is dominated by workers that have introverted personalities \cite{capretz2003personality}. Due to this, unstructured interviews would have probably been relatively ineffective because interviewees would have given only minimal input to the beginning question, and the discussion would have ended prematurely. On the other hand, structured interviews do not allow deviations from the interview questions, which may have left some exciting aspects unnoticed and limited the depth of the data \cite{alsaawi2014critical}. Due to these facts as a good compromise, semi-structured interviews were selected as interviewing method.

\subsection{Survey}
According to \citeauthor{groves2011survey}, a survey is a research methodology that gathers information from the sample population to form quantitative knowledge of the attributes of the larger population \cite{groves2011survey}. Alternatively, it can be stated that a survey describes a population using counts and describes "what is out there" \cite{sapsford2006survey}. So, the survey produces a quantitative view of subjects handled in the survey from the perspective of a specific population.

At the beginning of this thesis, a survey was not considered for the research. A survey was not considered because it was assumed that interviews and other information collection methods would sufficiently provide knowledge to understand the case properly. However, during an interview session, it was suggested by the interviewee that the survey could be organised related to the research because he thought it could be essential to collect information also from the wider audience. This comment was stated outside the planned questions and concretely proved the power of the semi-structured interviews.

In the end, a survey was also included as a research method. Even though interviews can provide deep knowledge about the subject, there is a possibility that only some opinions and facts are heard during the interviews, which could leave some areas accidentally without investigation. The survey was selected as an information-gathering way to fill possible information holes to ensure that no accidental holes would be left during the research by involving a larger population.

\subsection{Literature review}\label{subsection:literature_review}
The literature review available in the \autoref{chapter:background} was done to understand the project's context and gather information about prior research. Because the selected field is large overall, the academic literature review focused only on the most important subjects. In addition to academic papers, other literature sources were also used for the information gathering described later in the \autoref{chapter:literature_reviews}. A literature review was selected as a research method because it is a common academic practice and provides a better understanding of the context of the subject under research.

\section{Success metrics}\label{success_metrics}
While hard metrics like coverage increase in a specific time frame, the percentage of the flaky automation test runs and the decrease of the regression bugs would have been a perfect fit for this thesis. Unfortunately, due to time constraints, any of the mentioned options were not realistic. Overall the project is long, which causes the effects of this project to be visible properly after an even longer time. Because writing the thesis should take about half a year, any of the mentioned metrics were not considered.

Instead, another metric needed to be invented to finish the thesis at the appropriate time. Because it is hard to measure a system under construction with hard metrics, soft professional review metrics were selected instead. Soft success metrics aim to capture the satisfaction of the professionals involved in the project and chart the satisfaction of the developers developing the test automation test cases in the future. Practically the success was measured by organising semi-structured interviews with the questions available in the \autoref{appendix:review_interview_questions}. Additionally, an anonymous survey was organised with the same questions as in the interviews to ensure that the interviewees could also safely provide negative feedback regarding the project.

The project's absolute success metrics are impossible to capture using professional reviews. However, it is possible to have a good approximation of how well the project succeeded, which is better than no metrics. As stated earlier, the exact metrics cannot be used during this project. However, during a smaller continuation study, exact success metrics related to this project could be collected.