\chapter*{ABSTRACT}
\addcontentsline{toc}{chapter}{ABSTRACT}

Lappeenranta-Lahti Univeristy of Technology LUT

LUT School of Engineering Science

Degree Programme in Software Engineering

% Some extra space.
\hspace{1em}

Roni Juntunen

\hspace{1em}

\textbf{Introducing test automation for an existing software project}

Case: M-Files Hubshare

\hspace{1em}

Master’s thesis

% Print current year.
\the\year{}

% Automatically insert amount of total pages, figures, tables and appendices.
\getpagerefnumber{end_of_main_contents} pages,  \TotalValue{totalfigures} figures, \TotalValue{totaltables} tables and \TotalValue{appendixchapters} appendices

\begin{flushleft}
	Examiners: Associate professor Jussi Kasurinen (D.Sc.)\\
\hspace{19.5mm} Lecturer Erno Vanhala (D.Sc.)\\
\end{flushleft}

Keywords: introducing, existing project, test automation, software project management,

\vspace{-8pt}
\leftskip=19.5mm architecture, unit testing, integration testing, system testing, case study, survey, M-Files, Hubshare\\

\leftskip=0pt

\hspace{1em}

The thesis describes how test automation can be introduced for an existing software project. The research is done for the company and is based on a real industry case. During the thesis, the issue is examined from three perspectives: test automation introduction, management and technical architecture. Based on these three viewpoints holistic description of the realistic test automation system is provided. In addition, some extra information is provided in the thesis, for example, from the education and motivation points of view. At the beginning of the thesis, different interview and survey-based information-gathering methods are described, and collected information is presented. Based on the discovered knowledge, requirements of the test automation system are formed. Later, the collected information is shaped into coherent plans covering the thesis' different perspectives. In the end, it is concluded that the introduction of test automation is possible for the existing software project, assuming that the introduction project is planned well enough and sufficient resources for the project are provided by the management.