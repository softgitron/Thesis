\chapter*{TIIVISTELMA}

Lappeenrannan–Lahden teknillinen yliopisto LUT

LUT Teknis-luonnontieteellinen

Tietotekniikan Koulutusohjelma

\hspace{1em}

Roni Juntunen

\hspace{1em}

\textbf{Testiautomaation käyttöönottaminen olemassa olevaan ohjelmistoprojektiin}

Tapaus: M-Files Hubshare

\hspace{1em}

Diplomityö / Kandidaatin työ

\the\year{}

% Automatically insert amount of total pages, figures, tables and appendices.
\getpagerefnumber{end_of_main_contents} sivua,  \TotalValue{totalfigures} kuvaa, \TotalValue{totaltables} taulukkoa ja \TotalValue{appendixchapters} liitettä

\begin{flushleft}
	Tarkastajat:  Apulaisprofessori Jussi Kasurinen (TkT.)\\
	\hspace{20.0mm} Yliopisto-opettaja Erno Vanhala (TkT.)\\
\end{flushleft}

Avainsanat: käyttöönotto, nykyinen projekti, automaatiotestaus, ohjelmisto projektin hallinta,

\vspace{-8pt}
\leftskip=21.0mm arkkitehtuuri, yksikkötestaus, integraatiotestaus, järjestelmätestaus, tapaustutkimus, kyselytutkimus, M-Files, Hubshare\\

\leftskip=0pt

\hspace{1em}

Diplomityö kuvailee, kuinka testiautomaatio voidaan ottaa käyttöön meneillään olevassa ohjelmistoprojektissa. Tutkimus on tehty yritykselle ja työ perustuu aitoon tapaukseen ohjelmistoteollisuudessa. Ongelmaa lähestytään diplomityössä kolmesta näkökulmasta: testausautomaation käyttöönotto, hallinnointi ja tekninen arkkitehtuuri. Perustuen näihin kolmeen näkökulmaan, kokonaisvaltainen kuvaus todenmukaisesta testausautomaatio järjestelmästä tarjotaan työssä. Lisäksi työ sisältää myös joitakin muita yksityiskohtia, kuten muun muassa koulutukseen ja motivaatioon liittyviä seikkoja. Diplomityön alussa haastatteluihin ja kyselytutkimukseen perustuvat tiedonkeräys tavat avataan ja lisäksi kerätyt tulokset esitellään. Löydettyjen tietojen pohjalta muodostetaan testausautomaatiojärjestelmän vaatimukset. Myöhemmin kerätyt tiedot muunnetaan johdonmukaisiksi suunnitelmiksi, jotka pitävät sisällään työn eri näkökulmat. Lopuksi todetaan, että testausautomaation käyttöönotto on mahdollista nykyiselle ohjelmistoprojektille, olettaen että projekti on suunniteltu riittävän hyvin ja riittävät resurssit on taattu projektille hallinnon tasolta.