\chapter{DISCUSSION}\label{chapter:discussion}
This chapter will critically analyse and discuss different aspects of the project. The project as a whole will be discussed in the first two sections. After that, the overall impressions of the project's three research questions will be addressed. Finally, some remarks related to academic matters are left in the last two sections.

\section{Reflection}
This section includes a reflection of the work based on logical reasoning. The first subsection reflects the overall aspects of the project, while the later sections are divided based on the research questions.

\subsection{Overall impressions of the project}
Overall the project went smoothly, despite multiple local obstacles along the way. The project almost failed right at the beginning. As was mentioned in the \autoref{subsection:beginning_of_the_project}, the project's goals were first set among the \gls{qa} team members. Setting initial goals without the Hubshare team was an apparent mistake in hindsight because the original project goals were severely detached from reality. Fortunately, the project's goals were quickly redefined after meeting with the Hubshare team, but the project could have failed miserably if the original plans had been followed blindly.

Even though the planning process is presented in a nicely linear form in this thesis, the reality during the project was quite different. Instead of doing things linearly, all tasks were done in parallel. Parallel execution of the tasks was a hard requirement because the project began during the holiday season. For example, in the beginning, the interviews nor the survey could be arranged with reasonable participation rates due to vacations. It was necessary to do background research, design architecture and even carry out implementation simultaneously to fill the schedule appropriately from time to time.

Even though there were some spontaneous changes, in principle, the selected meta-planning, planning and execution strategy worked well. During the pre-planning phase, it was discovered and understood what the necessary components for the project were. In the planning phase, the understanding was turned into more detailed plans. Finally, those plans were refined when necessary during the execution phase and turned into tangible test automation systems. From the grander perspective, these phases follow the waterfall model's requirements, program design and coding phases \cite{royce1987managing} presented in the \autoref{subsection:test_automation_daily_activities}. However, due to the parallelism mentioned in the previous paragraph, the practical execution of the project still leaned heavily towards an agile way of working.

Selected research methods also worked well as part of the project. Background material research provided a foundation for the project through literature, internal documents and code reviews. Interviews provided in-depth details and hidden knowledge about the details of the existing processes and the test automation systems. Finally, the survey provided broader coverage to ensure that all aspects have been adequately considered. Using this extensive data collection methodologies, it is unlikely that any vital aspect could have slipped past unnoticed.

However, one planning aspect that could have been more successful during the project was requirements management. During the project, the requirements were, in fact, successfully collected using the methods mentioned in the previous paragraph, so the collection of the requirements was not the problem. The problem was that the requirements were not made available in a central place until the table in the \autoref{appendix:projects_requirements} was constructed. Fortunately, the lack of proper requirements management during the project did not cause any permanent damage, but there were a couple of close calls. If some essential requirements had been accidentally forgotten during the project, it would have led to catastrophic consequences by severally impacting the production of the artefacts by causing significant changes late in the production. That is also why requirements have been seen as a necessity in the literature \cite{hood2007requirements}.

Deeper reasons why the project was successful can be divided into two reasons. The project succeeded well because all aspects were rigorously extracted from various sources and based on the best available knowledge. Additionally, the project was successful because its status was well communicated, and all feedback was always considered while produced results were shared with outside peers when appropriate. Actually, due to results sharing, the project outcomes did have a significant effect beyond its intended goal by affecting processes company-wide, partly showing how well the project fared in the end.

\subsection{Management impressions}
Management plans were already so well refined during the first review round of several managers that hardly any comments were left on the original document. After some minor fixes management plan was swiftly approved. Based on the smoothness of the review round, it can be estimated that management plans were quite solid.

Even though the management plans were solid from the perspectives considered during the creation of the plans, there was still one usually critical aspect that was not addressed: an analysis of the costs. Generally, it is recommended that the costs of the automation are analysed during the management planning as mentioned in the \autoref{subsection:prioritization_of_the_test_automation}. Analysis of costs was not required during this project as mentioned in the \autoref{section:other_planning_aspects}, but it is still clearly an improvement that could have been made to the plans.

Because this thesis concentrated on the higher management and technical aspect, education and motivation-related matters were not adequately addressed in this thesis. However, during the actual project, also those aspects were fully addressed. During the data gathering, it became evident that education and motivational matters are crucial for the project's overall success. Even so crucial that without considering those aspects, the project would automatically fail because even though the produced test automation system would be perfect, it would only be helpful if someone knew how to use and develop the system and was motivated to do so. The importance of education and motivation has also been identified in other projects according to literature \cite{graham2012experiences}.

\subsection{Introduction plan impressions}
In the end, the introduction plans were relatively straightforward. All of the test automation systems would be built one at a time, from the simplest to the most complex and introduced with the education after the implementation is ready. Due to technical details, in practice, there were only a few possibilities to organise the introduction of the test automation systems. Because more complex systems always required the knowledge and techniques from the lower levels as presented in the \autoref{section:technical}, it was not easily possible to start, for example, with system testing automation. Some parts could have been swapped, like, for example, the implementation order of the integration test frameworks, but in the grander scale of things, that is insignificant because all of the test automation systems will be constructed in any case.

\subsection{Technical architecture impressions}
Especially frontend unit testing design is sound because it uses the built-in technologies of Angular and should work well because other Angular developers have already tested the framework. Backend unit testing design should also be good because it is based on popular tested techniques. However, the custom part of the framework may have some unanticipated issues in the future because the technology is new. These components were still necessary so that all the goals of the test automation system could be reached, so they were a calculated risk.

Integration frameworks use well-established technologies and safe methods for archiving their goal. It is unlikely that routine operating system procedures or database backup-restore mechanisms would cause any unexpected problems. Integration frameworks also achieve their goal nicely by using a common platform that consequently decreases the development efforts of the system.

Because the system testing framework is based on the integration testing framework, it shares the same positive traits. Additionally, the framework uses well-established technologies to achieve its goals, so the system will likely work well. Selected Gherkin-based test language should significantly increase the understandability of the system for the test case designer and, in the future, enable new kinds of test automation development scenarios by making the development more accessible.

In the literature, there are also other distinctive ways to do testing. For example, there is visual model-based testing, and \gls{ai} assisted testing techniques available, as was mentioned in the \autoref{chapter:introduction} and the \autoref{subsection:system_testing}. Even though these testing techniques are intriguing based on technical searching and, for example, the tool popularity graphs \cite{stateofjs2021}, neither of these techniques seems to be widely used, at least for now. Instead, more traditional methods are still dominating the practical implementations. Naturally, it would have been possible to experiment with more exotic methods, but this was not a feasible alternative during this study due to time constraints. It would have been risky to select more exotic methods as first choices because they have yet to be widely used and may lack critical features. Due to the mentioned factors, in the end, the more exotic methods were not researched during this study.

\section{External feedback}
This section will present how satisfied different parties were with the progress of the work and its outcomes. Evaluations were collected using semi-structured interviews and a survey as was described in the \autoref{success_metrics}. Interview questions can be found in the \autoref{appendix:review_interview_questions}. Sub-sections of this section are divided based on the interview question categories.

The collected feedback corresponds only to the state achieved during this thesis's construction. As described in the \autoref{section:goals_and_limitations}, only a small part of the whole system (namely backend unit test automation) was fully taken into use during this thesis. The feedback considers mainly parts of the project that were fully completed before the completion of this thesis.

\subsection{Satisfaction with the project}\label{subsection:satisfaction_with_the_project}
Overall, all interviewees mentioned the project's importance and praised the advancements done to test automation so far. The interviewees especially liked the self-direction of the project's leader and the project's good and transparent management with opaque weekly meetings. The project always had a direction and moved forward, despite challenges along the way. Only minimal effort was required from the rest of the team, which decreased resource usage. From the practical standpoint, interviewees especially liked the included training and constructed technical documentation.

Even though interviewees liked the self-direction and minimal effort from other team members, they also identified the problems with the selected approach. With a tighter project group, some of the knowledge about the technical specifics of the test automation system was not appropriately shared with other team members. Because of that, all of the interviewees agreed that the rest of the team should have provided more time for the project and been assigned tasks related to building the test automation system. Interviewees also agreed that getting started with the new test automation systems is more challenging without internal knowledge.

\subsection{Satisfaction with the management plans}
Regarding the management plans, interviewees especially liked the definition of ownership regarding test cases and clear concrete success indicators included in the plans. Additionally, the automation coverage improvement plan was considered to be good. The management plans were considered to be rigorous, regular and clear.

According to interviewees, the main problem with the test automation management plans was that the defined plans were not introduced to the developers nor taken into use. Not taking the management plans into use is problematic because, on paper alone, the plans cannot provide any benefits. Slower management adoption was probably partly caused by the lack of whole-team involvement. Fortunately, from a practical perspective, the project is still young, and the management plans can be adopted in the future. The change takes time, and it is not so easy to take new plans into use overnight as was mentioned in the \autoref{subsection:management_key_findings}.

\subsection{Satisfaction with the produced technological stack}\label{subsection:satisfaction_with_the_produced_technological_stack}
Interviewees commented that the technical challenges were identified well before building the test automation system so that the problems could be avoided during the implementation of the system. Additionally, reasonable compromises between usability and features were found by identifying the requirements early. The backend unit testing framework was described to be powerful and easy to use. Especially the mocking system was found to be powerful and flexible. Documentation of the system was generally praised, but some concerns were raised if the documentation is good and broad enough for future newcomers.

While the powerfulness of the backend unit testing framework was praised, according to interviewees, it was also one of its primary weaknesses. Due to the framework's powerfulness, its internal construction became complex, hindering its understandability. Complexity was not helped by the fact that some error codes produced by the framework are cryptic. Because of internal complexity, some interviewees also thought the system was complex to use. Also, there were some concerns regarding the future performance of the system. Part of the complexity impressions probably comes from the fact that the team was not sufficiently involved with building the framework as mentioned in the \autoref{subsection:satisfaction_with_the_project}. The framework's usage was sufficiently taught to the team, but the knowledge about the internal construction was not sufficiently provided.

\section{Answers to the research questions based on the work}
In the \autoref{rq1}, it was asked, "What management-related aspects must be considered regarding the test automation systems?". Based on the research done during this thesis, it can be concluded that at least the following management-related matters must be considered while planning test automation systems:
\begin{itemize}[noitemsep]
	\item Overall, how the test automation process is going to work? $\rightarrow$ As part of the normal agile development process.
	\item What responsibilities each stakeholder has related to automation testing? Who creates the test cases? $\rightarrow$ Developers. Who takes responsibility for the test automation framework development? $\rightarrow$ Separate test automation team.
	\item How much time and resources can be used for the test automation development?
	\item Which part of the code base should be automated?
	\item What are the common test automation conventions?
\end{itemize}
More management-related aspects in the form of requirements are available in the \autoref{appendix:projects_requirements}.

Additionally, as was learned during this project and discussed in the \autoref{subsection:satisfaction_with_the_project}, the whole team must be involved with the test automation fully from the beginning. The development silo is still a silo, even if it is made of glass, as encountered in this project. Sufficient management resources must be assured to avoid siloing effects inside the team. The importance of sufficient management resources has also been identified in the literature \autoref{subsection:test_automation_roles}.

In the \autoref{rq2}, it was asked: "How can test automation effectively be taken into daily use?". The success in this category seems to come down to developers' motivation and support. If the developers' do not have sufficient motivation and support, the test automation efforts will fail. The following things must be in good shape to improve these factors:
\begin{itemize}[noitemsep]
	\item Proper planning.
	\item Management support.
	\item Test automation architecture and tooling.
	\item Education of the developers.
\end{itemize}
Additionally, to improve the previously listed factors, there must be a clear plan for how these aspects are adequately considered. During this project, the educational aspects were, for example, addressed by including them directly in the introduction plan presented in the \autoref{subsection:test_automation_introduction_plan}.

\autoref{rq3}: "How to design a usable, performant and stable regression test automation system in all three major testing levels (unit, integration and system)?" concentrated on the more technical aspects of the test automation. From the architectural perspective, test automation systems generally follow good software production practices. In the beginning, the requirements of the system must be charted. After that, test automation systems architecture must be outlined, so it is clear how the system's production can be started. Finally, the development of the system can begin with constant adjustments of the development based on the feedback in the Agile fashion.

Developer ergonomics must always be considered when designing a usable test automation system. For example, it must be ensured that the test automation cases are easy to debug and that the mocking of the methods is as easy as possible. On the other hand performance of the test automation system can be primarily assured by the scalability of the test automation system, for example, by supporting multiprocessing in the test automation framework. Finally, the stability of the test automation system can be achieved by having sufficient isolation of the test cases using either developer rules or technical measures. Interestingly even though these aspects were raised multiple times during the survey as specified in the \autoref{subsection:technical_key_findings}, the aspects were not raised that actively anymore during the review interviews presented in the \autoref{subsection:satisfaction_with_the_produced_technological_stack}. The lack of mentions during the interviews infers that developers take these facts for granted, and when they work well, there is no need to mention these aspects.

\section{Academic analysis}
This thesis had the ambitious goal of including all the higher levels needed to introduce the comprehensive industrial-level test automation system. From the practical and project viewpoint, this worked great. However, from the academic point of view, the depth of this thesis could have been higher during certain sections. The partial lack of depth is an unfortunate side effect caused by the project's immense scope. The situation could have been avoided by reducing the project scope or increasing the number of pages of the thesis. Unfortunately, the true greatness of the scope became fully understood too late in the project, and the number of pages could not be increased indefinitely. Hence, some details had to be omitted so everything essential could be included between the covers.

\section{Work's generalizability}
Because this thesis concentrated only on a single project in a single company, it is hard to predict exactly how well this research's results are generalisable. Because every company is different, it is unlikely that the process and conventions described in this thesis would fit as is for other software companies. Additionally, especially technical architectures and solutions presented in this thesis are unlikely to fit other software projects directly because different software projects are technically notably different. The impossibility of absolute generalizability has also been identified in the literature for these kinds of qualitative studies \cite{marshall1996sampling}.

Even though most of the parts presented in this thesis cannot be used directly in other software projects, with a high probability, many of the ideas, ways of working, and technical choices are still partly reusable in other similar projects. For example, another project could reuse the introduction plan otherwise but could skip the system testing part because the system under testing does not have any \gls{ui}. From the technical side, someone could, for example, use the fast database reset but could skip the multiprocessing part because that would be too heavy to run for the particular application.